\documentclass{ltxdoc}

\EnableCrossrefs
\CodelineIndex
\RecordChanges

\usepackage{minted}
\usepackage{hyperref}


\begin{document}

\title{The \textsf{luatex-type-definitions} package}
\author{%
  Josef Friedrich\\%
  \url{josef@friedrich.rocks}\\%
  \href{https://github.com/Josef-Friedrich/LuaTeX_Lua-API}
       {github.com/Josef-Friedrich/LuaTeX\_Lua-API}%
}
\date{v0.2.0 from 2025/07/21}

\maketitle

\newpage

\tableofcontents

\newpage

% \section{Einführung}
\section{Introduction}

% \href{http://luatex.org}{Lua\TeX} verfügt über eine sehr umfangreiche
% \href{https://www.lua.org}{Lua} API. -->
\href{http://luatex.org}{Lua\TeX} has a very large
\href{https://www.lua.org}{Lua} API.
% Dieses Projekt versucht, diese API im Texteditor Ihrer Wahl
% zugänglich zu machen.
This project tries to make this API accessible in the text editor
of your choice.
% Ermöglicht wird dies durch den
% \href{https://github.com/LuaLS/lua-language-server}{lua-language-server}
% – einen Server, der das
% \href{https://en.wikipedia.org/wiki/Language_Server_Protocol}{Language
% Server Protocol (LSP)} für die Sprache Lua implementiert.
This is made possible by the
\href{https://github.com/LuaLS/lua-language-server}{lua-language-server}
- a server that implements the
\href{https://en.wikipedia.org/wiki/Language_Server_Protocol}{Language
Server Protocol (LSP)} for the |Lua| language.
% Funktionen wie die Syntaxhervorhebung, Code-Vervollständigung und die
% Markierung von Warnungen und Fehlern sollten daher nicht nur in
% \href{https://code.visualstudio.com}{Visual Studio Code}, sondern in
% einer \href{https://langserver.org/#implementations-client}{großen Anzahl an
% Editoren} möglich
% sein, die das „LSP“ unterstützen.
Features such as code completion, syntax highlighting and marking of
warnings and errors, should therefore not only be possible in
\href{https://code.visualstudio.com}{Visual Studio Code}, but in a
\href{https://langserver.org/#implementations-client}{large number of
editors} that support the |LSP|.

\section{Version on CTAN}

% Die Typ-Definitionen werden auf CTAN als eine einzige Datei
% veröffentlicht, um das CTAN-Verzeichnis nicht mit vielen einzelnen
% Lua-Dateien zu überladen.
The type definitions are published on CTAN as a single file to avoid
cluttering the CTAN directory with many individual Lua files.
% Da diese eine Datei knapp 1,5 MB groß ist, muss eine Konfiguration
% vorgenommen werden, damit der Sprachserver die Datei laden kann.
Since this one file is just under 1.5 MB in size, a configuration must
be made so that the language server can load the file.
% Das folgende Konfigurationsbeispiel legt die Dateigröße für das Laden
% von Typdefinitionen auf maximal 5000 kB fest.
The following configuration example sets the preload file size to a
maximum of 5000 kB.

\begin{minted}{json}
{
  "Lua.workspace.preloadFileSize": 5000,
}
\end{minted}


% Es gibt mehrere Möglichkeiten die Typdefinitionen in einem Projekt
% einzubingen.
There are several ways to include the type definitions in a project.
% Am einfachsten ist es, die Datei in den Projektordner zu kopieren.
The easiest way is to copy the file into the project folder.
% Oder Sie verwenden die Konfiguration |Lua.workspace.library|
Or you can use the configuration |Lua.workspace.library|:

\begin{minted}{json}
{
    "Lua.workspace.library": ["/path/to/luatex-type-definitions.lua"]
}
\end{minted}

\pagebreak
\PrintIndex
\end{document}
