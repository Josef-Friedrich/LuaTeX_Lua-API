\documentclass{ltxdoc}

\EnableCrossrefs
\CodelineIndex
\RecordChanges

\usepackage{minted}
\usepackage{hyperref}


\begin{document}

\title{The \textsf{luatex-type-definitions} package}
\author{%
  Josef Friedrich\\%
  \url{josef@friedrich.rocks}\\%
  \href{https://github.com/Josef-Friedrich/LuaTeX_Lua-API}
       {github.com/Josef-Friedrich/LuaTeX\_Lua-API}%
}
\date{v0.2.0 from 2025/07/21}

\maketitle

\newpage

\tableofcontents

\newpage

% \section{Einführung}
\section{Introduction}

% \href{http://luatex.org}{Lua\TeX} verfügt über eine sehr umfangreiche
% \href{https://www.lua.org}{Lua} API. -->
\href{http://luatex.org}{Lua\TeX} has a very large
\href{https://www.lua.org}{Lua} API.
% Dieses Projekt versucht, diese API im Texteditor Ihrer Wahl
% zugänglich zu machen.
This project tries to make this API accessible in the text editor
of your choice.
% Ermöglicht wird dies durch den
% \href{https://github.com/LuaLS/lua-language-server}{lua-language-server}
% – einen Server, der das
% \href{https://en.wikipedia.org/wiki/Language_Server_Protocol}{Language
% Server Protocol (LSP)} für die Sprache Lua implementiert.
This is made possible by the
\href{https://github.com/LuaLS/lua-language-server}{lua-language-server}
- a server that implements the
\href{https://en.wikipedia.org/wiki/Language_Server_Protocol}{Language
Server Protocol (LSP)} for the |Lua| language.
% Funktionen wie die Syntaxhervorhebung, Code-Vervollständigung und die
% Markierung von Warnungen und Fehlern sollten daher nicht nur in
% \href{https://code.visualstudio.com}{Visual Studio Code}, sondern in
% einer \href{https://langserver.org/#implementations-client}{großen Anzahl an
% Editoren} möglich
% sein, die das „LSP“ unterstützen.
Features such as code completion, syntax highlighting and marking of
warnings and errors, should therefore not only be possible in
\href{https://code.visualstudio.com}{Visual Studio Code}, but in a
\href{https://langserver.org/#implementations-client}{large number of
editors} that support the |LSP|.

\pagebreak
\PrintIndex
\end{document}
